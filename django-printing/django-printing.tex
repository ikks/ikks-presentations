\documentclass[10pt,xcolor={usenames,dvipsnames}]{beamer}
\setbeamercovered{transparent}
\usepackage[utf8]{inputenc}
\usepackage[english]{babel}
\usetheme{Warsaw}
\usecolortheme{fly}
 
\usepackage{minted}
\usemintedstyle{monokai}

\begin{document}
\definecolor{darkolive}{RGB}{3,17,7}
\setbeamercolor{author}{fg=SkyBlue}
\setbeamercolor{date}{fg=white}
\title{Impresión en Django}   
\author{Igor Támara} 
\date{Diciembre 4 de 2014} 

\setbeamercolor{normal text}{fg=white,bg=darkolive}
\setbeamercolor{structure}{fg=white}

\setbeamercolor{alerted text}{fg=red!85!black}

\setbeamercolor{item projected}{use=item,fg=white,bg=item.fg!35}

\setbeamercolor*{palette primary}{use=structure,fg=structure.fg}
\setbeamercolor*{palette secondary}{use=structure,fg=structure.fg!95!black}
\setbeamercolor*{palette tertiary}{use=structure,fg=structure.fg!90!black}
\setbeamercolor*{palette quaternary}{use=structure,fg=structure.fg!95!black,bg=black!80}

\setbeamercolor*{framesubtitle}{fg=white}

\setbeamercolor*{block title}{parent=structure,bg=black!60}
\setbeamercolor*{block body}{fg=black,bg=black!10}
\setbeamercolor*{block title alerted}{parent=alerted text,bg=black!15}
\setbeamercolor*{block title example}{parent=example text,bg=black!15}

\frame{\titlepage} 

\frame{\frametitle{Table of contents}\tableofcontents} 

\section{Algunas Aproximaciones}

\subsection{Formatos preimpresos}

\begin{frame}[fragile]
\frametitle{pdfrw}
\begin{minted}{python}
import pdfrw

...
\end{minted}
\end{frame}

\subsection{Aprovechando capacidades del navegador}

\begin{frame}[fragile]
\frametitle{CSS}
\begin{minted}{css}
@media print {
    header nav, footer {
        display: none;
    }

...

}
\end{minted}
\end{frame}

\begin{frame}

Dos imágenes para comparar

\end{frame}

\subsection{De html a PDF}

\begin{frame}

Imagen de perro o gato durmiendo

\end{frame}

\begin{frame}
\frametitle{wkhtmltopdf}
Enumerar características

\end{frame}

\begin{frame}

Shhhh!!!!!

\end{frame}

\begin{frame}
\frametitle{\LaTeX}

Poner algunas cosas bonitas de matemáticas, incluído mostrar pdflatex y el leoncito

\end{frame}

\begin{frame}[fragile]
\frametitle{Armando con \LaTeX}
\begin{minted}{python}
import os

...
\end{minted}
\end{frame}


\section{ReportLab}

\begin{frame}[fragile]
\frametitle{Puro y duro}
\begin{minted}{python}
import reportlab

 Hello world
...
\end{minted}
\end{frame}

\begin{frame}

Otro animal durmiendo

\end{frame}

\begin{frame}
\title{xhtml2pdf}

Mostrar características y contar que hay alguna estilización con CSS

\end{frame}

\section{RML}

\begin{frame}

Imagen de algo muy feo, que repugna

\end{frame}

\begin{frame}
\frametitle{RML con buena documentación}

Contar que hay muy buena documentación y hacer publicidad a la empresa

\end{frame}


\subsection{z3c.rml}

\begin{frame}[fragile]
\frametitle{Inserting source code}
\begin{minted}{python}
import numpy as np
 
def incmatrix(genl1,genl2):
    m = len(genl1)
    n = len(genl2)
    M = None #to become the incidence matrix
    VT = np.zeros((n*m,1), int)  #dummy variable
 
    #compute the bitwise xor matrix
    M1 = bitxormatrix(genl1)
    M2 = np.triu(bitxormatrix(genl2),1) 
...
\end{minted}
\end{frame}

\section{Otras aproximaciones}
\subsection{LibreOffice.org}

\begin{frame}

Imagen de Godzilla y el teso del Amazonas.

\end{frame}


\begin{frame}
\frametitle{Relatorio}

This is offered as is

\end{frame}


\section{PQR}

\begin{frame}
\frametitle{Sin garantías}

This is offered as is

\end{frame}

\end{document}
